\documentclass{article}
\usepackage{amsmath}
\usepackage{xspace}
\usepackage[amssymb]{SIunits}
\addunit{\molar}{M}
\newcommand{\ICa}{\ensuremath{I_\mathrm{Ca}}\xspace}
\newcommand{\gCa}{\ensuremath{g_\mathrm{Ca}}\xspace}
\newcommand{\ECa}{\ensuremath{E_\mathrm{Ca}}\xspace}
\newcommand{\Cmem}{\ensuremath{C_\mathrm{m}}\xspace}
\newcommand{\dif}{\textrm{d}}


\begin{document}
\section{Test 1 - Ca accumulation}
\label{tests:sec:test-1-ca}

1 compartment, length $L$, diameter $d$, membrane capacitance \Cmem,
membrane potential $V$. Calcium current $\ICa=\gCa(V-\ECa)$, where $g$ is
conductance and $\ECa$ is calcium reversal potential. $g$ is 0 apart
from when it is set to $\overline{g}$ from $t_1$ to $t_2$.

ODEs describing membrane potential and calcium concentration:
\begin{equation}
  \label{tests:eq:1}
  \Cmem\frac{\dif V}{\dif t} = \ICa
\end{equation}
 This can be solved:
\begin{equation}
  \label{tests:eq:2}
  V(t) = V(t_0) + (\ECa - V(t_0))(1-\exp((t-t_0)\gCa/\Cmem))
\end{equation}


The equation describing the calcium concentration [Ca] is:
\begin{equation}
  \frac{\dif [\mathrm{Ca}]}{\dif t} = \frac{\ICa a }{2Fv}
\end{equation}
where $a= \pi Ld$ is the surface area and $v = \pi Ld^2/4$ is the
volume. 

Hence following should be true:
\begin{equation}
[\textrm{Ca}](t_1) - [\textrm{Ca}](t_0) =  \frac{\Cmem a (V(t_1) -
    V(t_0))  }{2Fv} =  \frac{2\Cmem  (V(t_1) -
    V(t_0))  }{Fd}
\end{equation}

\begin{tabular}{lll}
Quantity       & Symbol & Units \\
\hline
Voltage        & $V$    & mV    \\
Time           & $t$    & ms    \\
Conductance    & $g$    & \siemens\per\centi\square\meter \\
Concentration  & [Ca]   & mM \\
Faraday const. & $F$    & C\per\mole  \\
Length         & $l$    & \micro\meter \\
Diameter       & $d$    & \micro\meter \\
Capacitance    & $\Cmem$ & \micro\farad\per\centi\square\meter \\
\hline
\end{tabular}

Hence the factor-label method gives: 
\begin{equation}
  \begin{split}
  \frac{\micro\farad}{\centi\square\meter}\cdot\frac{\micro\meter\squared\cdot\milli\volt}%
  {\coulomb\per\mole\cdot\micro\meter\cubed}
  = \frac{10^{-6} \farad}{10^{-4}\square\meter}\cdot\frac{\mole\cdot10^{-3}\volt}%
  {\coulomb\cdot10^{-6}\meter}
  = \frac{10^{-6} \coulomb}{10^{-4}\volt\square\meter}\cdot\frac{\mole\cdot10^{-3}\volt}%
  {\coulomb\cdot10^{-6}\meter}\\
  = \frac{10^1\mole}{\meter\cubed} =
  \frac{10^1\mole}{10^3\deci\meter\cubed} = 10^{-2}\molar = 10\milli\molar
  \end{split}
\end{equation}
  


\end{document}
